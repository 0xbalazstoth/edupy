\section{Indexelés}

\begin{quote}
    Az indexelés a szövegekben és listákban való elemek elérésére szolgál.
Az indexelés 0-tól kezdődik, azaz az első elem indexe 0, a másodiké 1, és így tovább.
Az indexelés negatív számokkal is lehetséges, ahol az utolsó elem indexe -1, az utolsó előttié -2, és így tovább.
\end{quote}

\subsection{Pozitív indexelés}
\begin{pycode}
    text = "Hello, World!"
    first_character = text[0]
    second_character = text[1]
    third_character = text[2]
    last_character = text[-1]
    second_last_character = text[-2]
\end{pycode}

\subsection{Negatív indexelés}
\begin{pycode}
    text = "Hello, World!"
    last_character = text[-1]
    second_last_character = text[-2]
\end{pycode}

\subsection{Szeletelés}
\begin{pycode}
    text = "Hello, World!"
    first_word = text[:5]
    second_word = text[7:]
    middle_word = text[5:12]
\end{pycode}