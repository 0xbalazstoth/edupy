\section{Primitív adattípusok}
\begin{quote}
    Másnéven, előre meghatározott (alapvető) adattípusok.
\end{quote}
\begin{quote}
    Értékadás szintaxisa: \texttt{<változó> = <érték>}
\end{quote}

\begin{itemize}
    \item Szöveg (Text): str (string)
    \item Szám (Number): int, float, complex
    \item Logikai (Boolean): bool
    \item Értékhiány (None): NoneType
\end{itemize}

\subsection{Szöveg (str)}
\begin{pycode}
    greeting = "Helló"
    name = "XYZ"
\end{pycode}
\begin{minted}{pycon}
>>> greeting = "Helló"
>>> name = "XYZ"
\end{minted}

\subsubsection{Karakter}
\begin{pycode}
    character = "X"
\end{pycode}
\begin{minted}{pycon}
>>> character = "X"
\end{minted}

\subsubsection{String-ek összefűzése}
\begin{pycode}
    greeting_user = greeting + ", " + name
\end{pycode}
\begin{minted}{pycon}
>>> greeting_user = "Helló, XYZ"
\end{minted}

\subsubsection{Ugyanazon érték megadása több változónak egyszerre}
\begin{pycode}
    friend_name = "XYZ
    friend1 = friend2 = friend3 = friend_name
\end{pycode}
\begin{minted}{pycon}
>>> friend_name = "XYZ"
>>> friend1 = friend2 = friend3 = "XYZ"
\end{minted}

\subsubsection{Többsoros string}
\begin{pycode}
    multi_line_string = """Ez egy
    többsoros
    string"""
\end{pycode}
\begin{minted}{pycon}
>>> multi_line_string = """Ez egy
... többsoros
... string"""
\end{minted}

\subsubsection{Alapvető string műveletek}
\begin{pycode}
message = "Ez egy üzenet!"
upper_message = message.upper() # Nagybetűssé alakítás
lower_message = upper_message.lower() # Kisbetűssé alakítás
remove_whitespace = message.strip() # VEZETŐ és VÉG szóközök eltávolítása
replace_message = message.replace("üzenet", "szöveg") # Csere
\end{pycode}
\begin{minted}{pycon}
>>> message = "Ez egy üzenet!"
>>> upper_message = "EZ EGY ÜZENET!"
>>> lower_message = "ez egy üzenet!"
>>> remove_whitespace = "Ez egy üzenet!"
>>> replace_message = "Ez egy szöveg!"
\end{minted}

\subsubsection{Bemenet}
\begin{quote}
    \texttt{input()} függvény: A felhasználótól vár bemenetet, és azt string-ként adja vissza.
\end{quote}
\begin{pycode}
    user_input = input("Kérem, adja meg a nevét: ")
    greeting_message = greeting + ", " + user_input
\end{pycode}
\begin{minted}{pycon}
>>> user_input = input("Kérem, adja meg a nevét: ")
Kérem, adja meg a nevét: XYZ
>>> greeting_message = "Helló, XYZ"
\end{minted}

\subsection{Szám (Number)}
\subsubsection{Egész szám (int)}
\begin{pycode}
    integer_number = 10
\end{pycode}
\begin{minted}{pycon}
>>> integer_number = 10
\end{minted}

\subsubsection{Lebegőpontos szám (float)}
\begin{pycode}
    float_number = 10.5
\end{pycode}

\begin{minted}{pycon}
>>> float_number = 10.5
\end{minted}

\subsubsection{Komplex szám (complex)}
\begin{quote}
    \texttt{a + bj}, ahol \texttt{a} és \texttt{b} valós számok, \texttt{j} pedig az imaginárius egység.
\end{quote}
\begin{pycode}
    complex_number = 1 + 2j
\end{pycode}
\begin{minted}{pycon}
>>> complex_number = 1 + 2j
\end{minted}

\subsubsection{Aritmetikai műveletek}
\begin{pycode}
    addition = 10 + 5 # Összeadás
    subtraction = 10 - 5 # Kivonás
    multiplication = 10 * 5 # Szorzás
    division = 10 / 5 # Osztás
    modulo = 10 % 5 # Maradékos osztás
\end{pycode}
\begin{minted}{pycon}
>>> addition = 15
>>> subtraction = 5
>>> multiplication = 50
>>> division = 2.0
>>> modulo = 0
\end{minted}

\subsubsection{Értékadó operátorok}
\begin{pycode}
    number = 10
    number += 5 # number = number + 5
    number -= 5 # number = number - 5
\end{pycode}
\begin{minted}{pycon}
>>> number = 10
>>> number += 5
>>> number -= 5
\end{minted}

\clearpage
\subsection{Logikai (Boolean)}
\begin{quote}
    Két értéke lehet: \texttt{True} (1) vagy \texttt{False} (0).
\end{quote}

\subsubsection{Igazságtábla}
\begin{minipage}{0.5\textwidth}
    \begin{center}
        \begin{tabular}{|c|c|c|}
            \hline
            \textbf{A} & \textbf{B} & \textbf{A ÉS B} \\
            \hline
            1 & 1 & 1 \\
            1 & 0 & 0 \\
            0 & 1 & 0 \\
            0 & 0 & 0 \\
            \hline
        \end{tabular}

        \begin{quote}
            \texttt{A ÉS B} csak akkor igaz, ha \texttt{A} és \texttt{B} is igaz.
        \end{quote}
    \end{center}
\end{minipage}
\begin{minipage}{0.5\textwidth}
    \begin{center}
        \begin{tabular}{|c|c|c|}
            \hline
            \textbf{A} & \textbf{B} & \textbf{A VAGY B} \\
            \hline
            1 & 1 & 1 \\
            1 & 0 & 1 \\
            0 & 1 & 1 \\
            0 & 0 & 0 \\
            \hline
        \end{tabular}
    \end{center}
    \begin{quote}
        \texttt{A VAGY B} akkor igaz, ha legalább az egyik igaz.
    \end{quote}
\end{minipage}

\subsubsection{Összehasonlítás (érték alapján)}
\begin{pycode}
    is_equal = 10 == 10
    is_not_equal = 10 != 5
    is_greater = 10 > 5
    is_less = 10 < 5
    is_greater_or_equal = 10 >= 10
    is_less_or_equal = 10 <= 5
\end{pycode}
\begin{minted}{pycon}
>>> is_equal = True
>>> is_not_equal = True
>>> is_greater = True
>>> is_less = False
>>> is_greater_or_equal = True
>>> is_less_or_equal = False
\end{minted}

\clearpage
\subsubsection{Logikai műveletek}
\begin{quote}
    Lásd: Igazságtábla.
\end{quote}
\begin{pycode}
number = 5
and_result = number == 5 and number != 2  # True, ha mindkettő igaz
or_result = number == 10 or number == 5  # True, ha legalább az egyik igaz
not_result = not True  # True, ha az érték hamis
\end{pycode}
\begin{minted}{pycon}
>>> and_result = True
>>> or_result = True
>>> not_result = False
\end{minted}

\subsubsection{Objektumok összehasonlítása}
\begin{pycode}
x = 9
y = x
is_result = x is y
is_not_result = x is not y
\end{pycode}
\begin{minted}{pycon}
>>> is_result = True
>>> is_not_result = False
\end{minted}

\subsubsection{Tartalmazás vizsgálata}
\begin{pycode}
    text = "Hello, World!"
    is_in = "Hello" in text
    is_not_in = "Python" not in text
\end{pycode}

\subsection{Hogyan tárolódnak a memóriában a változók?}
\begin{minipage}{0.5\textwidth}
\begin{pycode}
value = "Hello, World!"
print(hex(id(value)))
print(type(value))

print("\n")

value = 10
print(hex(id(value)))
print(type(value))

other_value = value
print(hex(id(other_value)))
print(type(other_value))

value = 11
print(hex(id(value)))
print(type(value))

print("\n")
del value  # Névtérből törlődik csak
# print(value)
\end{pycode}
\end{minipage}
\begin{minipage}{0.5\textwidth}
\customwidthimage{2_0_variables_memory}{10cm}
\end{minipage}