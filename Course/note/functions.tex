\section{Függvények}

\begin{quote}
    A függvények olyan kód blokkok, melyeket újrahasználhatunk.
    A függvények segítségével csoportosíthatjuk a kódot, és egyszerűbben olvashatóvá tehetjük azt.
\end{quote}

\subsection{WET (Write Everything Twice) vs. DRY (Don't Repeat Yourself)}

\begin{quote}
    A WET (Write Everything Twice) és a DRY (Don't Repeat Yourself) két különböző programozási elv.
    A WET azt jelenti, hogy minden kódot kétszer vagy többször is megírunk, míg a DRY azt jelenti, hogy a kódot egyszer írjuk meg, majd újrahasználjuk.
\end{quote}

\subsection{Függvény létrehozása}
\begin{pycode}
    def greet():
       print("Hello, World!")
\end{pycode}


\subsection{Függvény meghívása}
\begin{pycode}
    greeting = greet()
\end{pycode}

\subsection{Paraméterek}
\begin{pycode}
    def greet(name):
        return f"Hello, {name}!"
\end{pycode}

\newpage
\subsection{Paraméterek alapértelmezett értékkel}
\begin{pycode}
    def greet(name="World"):
        return f"Hello, {name}!"

    greeting = greet()
    greeting_with_name = greet("Alice")
\end{pycode}

\subsection{Visszatérési érték}
\begin{pycode}
    def add(a, b):
        return a + b

    result = add(5, 3)
\end{pycode}

\subsection{Visszatérési érték nélkül}
\begin{pycode}
    def greet(name):
        print(f"Hello, {name}!")

    greet("Alice")
\end{pycode}